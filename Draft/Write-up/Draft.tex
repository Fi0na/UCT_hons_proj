\documentclass[12pt,a4paper]{article}

\usepackage{url}
\usepackage{appendix}
\usepackage[british]{babel}
\usepackage{amsmath}
\usepackage{hyperref}
\usepackage{graphicx}
\numberwithin{equation}{section}
\numberwithin{figure}{section}
\numberwithin{table}{section}

\usepackage[round]{natbib}
\bibliographystyle{natbib}
\def\bibsection{\section*{References}}

% Wrap around which gives all figures included the [H] command, or places it "here". This can be tedious to code in Rmarkdown.
\usepackage{float}
\let\origfigure\figure
\let\endorigfigure\endfigure
\renewenvironment{figure}[1][2] {
    \expandafter\origfigure\expandafter[H]
} {
    \endorigfigure
}

\let\origtable\table
\let\endorigtable\endtable
\renewenvironment{table}[1][2] {
    \expandafter\origtable\expandafter[H]
} {
    \endorigtable
}

\def\tightlist{}

\bibpunct[:]{(}{)}{;}{a}{,}{,}

\renewcommand{\baselinestretch}{1.5}

\begin{document}

\begin{titlepage}

\begin{center}
{\Huge \bf A comparison between the latest Prophet forecasting model and the ARIMA
model: An evaluation of BTC/ZAR predictability}\\
\today\\
Fiona Ganie (GNXFIO001)\\
{\tt GNXFIO001@myuct.ac.za}
\end{center}

\begin{abstract}
The Autoregressive Integrated Moving Average (ARIMA) model is one of the
most widely used time series models that has gained popularity in the
exchange rate market due to its ease of implementation and tractability.
With the evolution of computational power, soft computing techniques
have since been used in exchange rate markets. The ARIMA model has been
used as the standard benchmark model against which these complex methods
have been compared. Although they have produced significantly better
results than the ARIMA model, these models lack interpretability and
building them is a challenging task. Prophet is a sophisticated model
that differs from traditional time series models in that it can be
utilised by non-experts who have little knowledge about the statistical
intricacies involved in the model, however have domain knowledge about
the data generating process. Prophet is also robust to outliers and can
handle missing values in the time series without the need for
interpolation. This paper compares the out-of-sample forecasts of
Prophet and ARIMA over varying forecast horizons, using the Model
Confidence Set and Mincer-Zarnowitz test to evaluate forecast accuracy.
A complete dataset of the prices of Bitcoin/Rand as well as a dataset
which has missing prices and outliers present is used in the analysis.
It is found that the ARIMA model ranks ahead of Prophet over all
forecast horizons and is more robust when faced with missing values and
outliers. In addition, both models having equal predictive accuracy when
forecasting at short and long forecast horizons when the data is clean
and at short forecast horizons when the data is dirty, with unbiased
and/or efficient forecasts only generated over the longest forecast
horizon in the case of the clean data.
\noindent\\
Keywords: Prophet, ARIMA, Mincer-Zarnowitz,   Box-Jenkins, Bitcoin, Model Confidence Set
\end{abstract}
\end{titlepage}

\pagenumbering{arabic}

\section{\texorpdfstring{Introduction
\label{Introduction}}{Introduction }}\label{introduction}

The Autoregressive Integrated Moving Average (ARIMA) model is one of the
most widely used time series models that has attracted attention in
financial market forecasting (Khashei, Bijari, and Ardali 2009).
Although the Random Walk model has typically been applied to foreign
exchange markets and has produced superior results, some researchers
have contended that foreign exchange markets are not efficient and
believe that future prices depend on current and past events
(Abu-Mostafa and Atiya 1996). This has led to the application of the
ARIMA model to exchange rate problems, where it has since gained
popularity due to its ease of implementation and tractability. The ARIMA
model is easy to use and yields good forecasts over short forecast
horizons, however the linearity of the ARIMA model fails to adequately
capture the non-linearity inherent in exchange rate data (Zhang and Hu
1998).

With the evolution of computational power, non-linear, soft computing
techniques have been proposed as a solution. The ARIMA model has been
used as the standard benchmark model against which these more complex
methods have been compared. Although they have produced significantly
better results than the ARIMA model, these models lack interpretability
and building them is a challenging task. The ARIMA model is tractable
and less computationally expensive. It has been used as the building
block for more advanced models and has provided the inspiration for
hybrid versions of the model which have been used in exchange rate
forecasting.

In December 2016, Facebook open-sourced their forecasting model Prophet.
Prophet differs from traditional time series models such as the ARIMA in
that it can produce high quality forecasts in a straightforward way.
Prophet is a sophisticated model that provides informative results,
however is configurable and easy to use. It can be utilised by
non-experts who have little knowledge about the statistical intricacies
involved in the model, however have domain knowledge about the data
generating process. This knowledge can easily be incorporated into the
model through its intuitively adjustable parameters (Taylor and Letham
2017). Furthermore, Prophet is robust to outliers and can handle missing
values in the time series without the need for interpolation (Taylor and
Letham 2017). This allows analysts without knowledge on how to
pre-process the data, to utilise the model without sacrificing
predictive accuracy. If Prophet produces forecasts that are as good as
the ARIMA model, it can be compared to more complex forecasting methods
that are less tractable and flexible.

This paper broadly aims to compare the forecasts produced by the
traditional ARIMA model and Prophet through an evaluation of the
exchange rate between Bitcoin and the Rand (BTC/ZAR), using the Model
Confidence Set and a Mincer-Zarnowitz approach of measuring forecast
accuracy. More specifically, this paper aims at:

\begin{enumerate}
\def\labelenumi{\arabic{enumi}.}
\item
  Comparing the out-of-sample forecasts of ARIMA and Prophet over
  varying forecast horizons.
\item
  Comparing the out-of-sample forecast performance with missing values
  and outliers present in the data.
\end{enumerate}

Section \ref{Background} will present the key findings from the major
works in which comparisons have been made to the ARIMA model in exchange
rate forecasting. Section \ref{Data} will provide a brief explanation
and motivation for the dataset used in this paper. Section
\ref{Forecasting Methods} explains and compares the formulation of the
ARIMA and Prophet model. Section \ref{Methodology} will describe the
methodology used in selecting the ARIMA and Prophet models, the Model
Confidence Set and the Mincer-Zarnowitz approach of measuring forecast
accuracy. The results from applying the ARIMA and Prophet model to the
dataset is presented in Section \ref{Results}. Finally, section
\ref{Discussion and Conclusions} discusses the results and its financial
implications in a statistical context.

\section{\texorpdfstring{Background
\label{Background}}{Background }}\label{background}

\subsection{Forecasting Exchange Rates with the ARIMA
model}\label{forecasting-exchange-rates-with-the-arima-model}

ARIMA models have commonly been used in financial forecasting and are
popular for observing stock prices and exchange rates due to its power
and statistical properties (C.-S. Lin, Chiu, and Lin 2012). The ARIMA
model is attractive as it is tractable and produces good short-term
forecasts when more than 100 observations are used (Tseng et al. 2001).
They have frequently been used as a benchmark to compare new forecasting
techniques that have emerged over time and have yielded satisfactory
results when predicting exchange rates. Nwankwo (2014) forecasted the
exchange rate between the Naira and dollar, using Akaike's Information
Criterion (AIC) as a measure of performance. Diagnostic testing revealed
that the ARIMA(1,0,0) model was the best fit for the data.

Although the ARIMA model has the advantage of ease of implementation and
flexibility, it fails to capture the non-linearity and volatility
present in exchange rate data. Over time, the ARIMA model has evolved to
cater for a wider variety of data and to compensate for some of its
shortcomings. Some of the most popular versions of the ARIMA model that
has been implemented in exchange rate forecasting is the Seasonal ARIMA
(SARIMA) and Fractional ARIMA (ARFIMA) model.

The SARIMA model was introduced to capture the periodic behaviour of
data and extends the ARIMA class by including a term for seasonal
differencing. Etuk, Wokoma, and Moffat (2013) modelled the Naira/CFA
Franc exchange rate which exhibited monthly seasonality using an
additive SARIMA model, to demonstrate that it can be a useful fit for
exchange rate data which displays seasonality. Their results showed that
the SARIMA model adequately described the variation in the exchange rate
series.

The ARFIMA model generalises the ARIMA model in that the degree required
to make the data stationary can assume any real value, and is no longer
restricted to the integer domain. The ARFIMA model has the ability to
capture the dependence between observations that are widely spread apart
in time (Cheung 1993). This makes the model parsimonious since it can
capture long memory in data as well as short term dynamics (Cheung
1993). Cheung (1993) fitted the ARFIMA model to examine five exchange
rates and found that there was strong evidence of long memory in the
exchange rate time series.

\subsection{Other techniques used to forecast exchange
rates}\label{other-techniques-used-to-forecast-exchange-rates}

Generalised autoregressive conditional heteroskedasticity (GARCH) models
were later developed due to the failure of ARIMA models to capture the
volatility in financial markets (Anastasakis and Mort 2009). Fahimifard
et al. (2009) applied a GARCH model to the Rial/USD and Rial/EUR to
compare the forecasts yielded by the GARCH and ARIMA model over varying
forecast horizons. Their results illustrated that the GARCH model
outperformed the ARIMA model when using the Root Mean Square Error
(RMSE), Mean Square Error (MSE) and Mean Absolute Deviation (MAD) as
performance criteria.

Over time, financial forecasting methods have moved away from linear
models like ARIMA and GARCH, to soft computing techniques. These complex
techniques are non-linear and can fit complex time series more easily
(Castillo and Melin 2002). Unlike the ARIMA model, soft computing
techniques do not impose structural assumptions on the model apriori
(Castillo and Melin 2002). Some of the most commonly used artificial
intelligence methods used to forecast exchange rate data are Neural
Networks and Fuzzy Logistic Systems.

Artificial Neural Networks (ANNs) mimic the structure of the brain and
have had many successful applications in forecasting exchange rates.
They are data driven, can adapt to non-stationary environments and can
approximate any continuous function (Khashei and Bijari 2011). In a
study done by Fahimifard et al. (2009), the ANN was found as an
effective way to improve the forecasts of exchange rates. Superior
results were produced when compared to the ARIMA and GARCH model using
the RMSE, MSE and MAD as a measure of performance.

Fuzzy Logistic Systems (FLSs) were initially developed to solve problems
involving linguistic terms and have successfully been used in financial
forecasting (Khashei, Bijari, and Ardali 2009). Fuzzy logic tries to
imitate human reasoning and the decision-making process and allows for
finer rather than discrete decisions to be provided. Santos, Costa, and
Santos Coelho (2007) investigated how well FLSs and ANNs perform
compared to the traditional ARMA and GARCH model. He examined the
forecasts of Brazilian exchange rate returns by considering different
frequencies of the series and comparing their one-step ahead forecasts.
By analysing accuracy statistics, he found that FLSs and ANNs achieved
higher returns based on the forecasts they produced. Similar results
were found by Khashei, Bijari, and Ardali (2009) when he analysed the
predictive capabilities of FLSs, ANNs, the traditional ARIMA model and a
Fuzzy ARIMA model.

Although ANNs have been broadly applied in financial forecasting, the
process of building them is a complex task and there is no consistent
method of design compared to the traditional Box-Jenkins ARIMA model.
The performance of ANNs is sensitive to many modelling factors such as
the number of input nodes included and the size of the training sample
chosen (Zhang and Hu 1998). Like ANNs, there is no systematic approach
for designing FLSs and they are only understandable when simple.
Although FLSs has the advantage over ARIMA models in that they can be
applied to data with few observations available, it yields acceptable
rather than accurate results and are more suitable for problems which do
not require high accuracy.

The traditional ARIMA model produces less superior forecasts than its
hybrid forms and other complex non-linear techniques, however its
forecasts are still satisfactory. It is a simple model that is easy to
implement and has a consistent method of model design and selection.
ARIMA models are more robust and efficient than complex structural
models in relation to short-run forecasting. The fact that they have
been used as the foundation for more advanced models and have commonly
been used as a benchmark for comparison, justifies it as a good starting
point to compare it to Facebook's forecasting method, Prophet, that was
recently released.

\subsection{Forecasting with Prophet}\label{forecasting-with-prophet}

The techniques that have been considered for exchange rate forecasting
thus far, require the analyst to have vocational knowledge about time
series. Prophet differs from traditional time series models in that it
is flexible and can be customised by a large number of non-experts who
have little knowledge about time series, however have domain knowledge
about the data generating process. Prophet allows for a large number of
forecasts to be produced across a variety of problems and consists of a
robust evaluation system that allows for a large number of forecasts be
evaluated and compared. This is Facebook’s definition of forecasting
at scale.

Prophet’s Bayesian approach to forecasting allows the analyst to
incorporate their expert knowledge into the model building process and
has produced significantly improved forecasts compared to the ARIMA
model. Taylor and Letham (2017) forecasted the number of events on
Facebook using Prophet. The time series was impacted by holidays, had
strong multi-period seasonality, and a piecewise trend. The forecasts
produced by Prophet were compared to common forecasting techniques such
as exponential smoothing, ARIMA, the seasonal naive, and the naive
model. While exponential smoothing and the seasonal naive model were
quite robust, the ARIMA forecasts were fragile. No model besides Prophet
accounted for the dips around holidays and the upward trend of the time
series towards later observations. If Prophet produces forecasts that
are as good as the ARIMA model when forecasting BTC/ZAR, it can be
compared to more complex forecasting methods that are less tractable
such as the hybrid models and machine learning techniques seen earlier.

\section{\texorpdfstring{Data \label{Data}}{Data }}\label{data}

The dataset used in this study consists of the daily closing prices of
BTC/ZAR over the period 24 January 2016 to 17 July 2017. This comprises
of a total of 541 trading days and was the chosen time period for
analysis due to constraints in obtaining data over a longer period. The
data was obtained from Bitcoincharts.

Bitcoin is of specific interest as it presents an interesting parallel
to traditional exchange rate markets. The cryptocurrency is built on a
decentralised system and as a result its value cannot be directly
influenced by a central authority (Fantazzini et al. 2016). In addition,
the Bitcoin market has attracted attention worldwide and is currently
the leading cryptocurrency, with awareness and adoption of the currency
growing over time (Fantazzini et al. 2016). Its novelty makes it a
highly volatile hence speculative market and provides an opportunity for
forecasting.

In this study, the analysis of BTC/ZAR forecasts are based on the daily
continuous log returns. The daily closing prices are transformed into
returns by taking the log difference at each time t and is calculated as
follows:

\[ r_t = ln(\frac{p_t}{p_{t-1}}) \] where \(p_t\) represents the closing
price and \(r_t\) represents the log return for time t = 1,2,.,T.

\section{\texorpdfstring{Forecasting Methods
\label{Forecasting Methods}}{Forecasting Methods }}\label{forecasting-methods}

\subsection{The ARIMA model}\label{the-arima-model}

The ARIMA model expresses the process \{\(y_t\)\} as a function of the
weighted average of past values of the process and lagged values of the
residuals. The weighted average of the past p values of the process
represents an autoregressive (AR) process of order p.~It feeds back past
values of the process into the current value, inducing correlation
between all lags of the process. The weighted average of the q lagged
residuals represents a moving average (MA) process of order q. The
purpose of mixing the MA process with the AR process is to reduce the
large number of past values required by AR processes and to control for
the autocorrelation which it creates between lagged values of the
process. The combination of the AR(p) and MA(q) process results in a
more parsimonious model, and forms a stationary autoregressive moving
average (ARMA(p,q)) process defined as:

\[ 
    y_t = c + \sum_{i = 1}^{p} \theta_{i} y_{t-i} + \sum_{i = 1}^{q} \phi_{i} e_{t-i}+ e_t \label{eq1} \\ \notag
\] where c is a constant and \{\(e_t\)\} is a white noise process with
zero mean and variance \(\sigma^2\).

The ARIMA(p,d,q) model generalises the ARMA model in that it includes
both stationary and non-stationary processes. The parameter d is the
degree of differencing required to render the process stationary. If d
is equal to zero the process is stationary and equivalent to an ARMA
model, and if d is strictly positive the process requires differencing
to make it stationary. The ARIMA model can be defined succinctly using
the backward shift operator B, which shifts the process back by one unit
of time, and is defined as \(By_t = y_{t-1}\). The ARIMA model has the
form (Hyndman and Athanasopoulos 2014):

\[ 
    (1-\sum_{i = 1}^{p} \theta_{i} B^i)(1-B)^dy_t = c + (1 + \sum_{i = 1}^{q} \phi_{i} B^i)e_t \label{eq2} \\ \notag 
\] where c is a constant and \{\(e_t\)\} is a white noise process with
zero mean and variance \(\sigma^2\).

\subsection{The Prophet model}\label{the-prophet-model}

Prophet is similar to a Generalized Additive Model (GAM) - an additive
regression model that consists of non-linear and linear regression
functions applied to predictor variables (Taylor and Letham 2017). The
decomposable model is of the form:

\[ 
    y(t) = g(t) + s(t) + h(t) + e_t
\] where the components of the model represent the growth, seasonality
and holiday respectively, and \(e_t\) is white noise.

Prophet, like the GAM, frames the forecasting problem as a curve fitting
exercise and uses backfitting to find the regression functions. This
allows for the model to be fitted quickly and missing values and large
outliers to be handled elegantly. The regression model also provides
model flexibility and allows the analyst to interactively change model
parameters (Taylor and Letham 2017). The growth component of Prophet may
be modelled as a linear or non-linear function of time. Linear growth is
modelled by a piecewise constant function while non-linear growth is
modelled similar to population growths which use a logistic growth model
(Taylor and Letham 2017). A time-varying upper limit may be specified
for logistic growth, at which point the forecasts will saturate. This
carrying capacity allows the analyst to incorporate their prior
knowledge about the maximum obtainable growth level such as the total
market or population size into the model. Prophet accounts for changes
in the trajectory of this trend by automatically detecting and selecting
changepoints in the data at which the growth rate is allowed to change.
These changepoints have a Laplace prior distribution placed on them and
its scale parameter may be used to adjust the flexibility of the trend
and to choose how aggressively the model should follow historical trend
changes. Analysts may also adjust the number of potential changepoints
included or manually specify their location. This allows non-experts
with knowledge about events that may affect the growth rate to use the
parameter as a knob to either increase or decrease the number of
changepoints included (Taylor and Letham 2017). It also allows for the
analyst to add changepoints which the automatic selection procedure may
have missed or remove changepoints when the model is overfitting
historical trends (Taylor and Letham 2017).

The decomposable form of the model allows for multiple seasonality
components with different periods to be added to the model. Seasonality
components are modelled by a Fourier series and have a Normal prior
distribution placed on its parameters (Taylor and Letham 2017). The
spread parameter can be adjusted by analysts to smooth the seasonality
and change how much of historical seasonality is projected into the
future (Taylor and Letham 2017).

The analyst may also provide a list of important events and holidays
which have impacted the time series in the past or which they know might
impact it in the future. The list could include the name, date, and
country in which they have taken place or are expected to take place
(Taylor and Letham 2017). By specifying the country of occurrence,
separate lists can be populated for global events and holidays, and
country-specific events and holidays. The union of the global and
country-specific lists can then be used for forecasting. Like
seasonality, a Normal prior distribution is placed on the parameters of
the holiday component, and the scale parameter can be adjusted by
analysts to smooth the holidays (Taylor and Letham 2017).

Prophet’s ability to forecast at scale enables it to model a wide
variety of data and may be able to adequately fit exchange rate data.
Its non-linear components could capture the non-linearity present in
exchange rates. In contrast, the ARIMA model is a linear function of
previous observations and lagged residuals and fails to capture the
non-linearity inherent in exchange rate data. Prophet performs well on
data with strong multiple “human scale” seasonalities and historical
trend changes. This differs to the ARIMA model which requires the data
to be de-trended and the variance stabilised before the model can be
fitted. Hence Prophet could model the weekly seasonality of closing
prices due to low trading activity which occurs around the weekend and
high trading activity which occurs mid-week.

Choosing the correct combination of parameters for the ARIMA model is a
challenging task due to the array of possible choices. Although the
auto.arima function in R may be used to automatically select an ARIMA
model that best fits the data, completely automatic forecasting methods
are too brittle and do not allow for useful assumptions to be
incorporated into the model. Prophet makes use of a semi-automatic
forecasting technique that keeps the analyst-in-the-loop. Its default
settings are said to generate forecasts that are as accurate as those
produced by skilled forecasters. If the forecasts produced are
unsatisfactory, they can be improved by the analyst by configuring the
model through its easily interpretable parameters. Furthermore,
non-experts who have domain knowledge about factors that affect Bitcoin
or if the dates of events which could impact the price of Bitcoin are
known, it may be incorporated into the model by the analyst. Prophet can
produce forecasts over irregular time intervals and allows for missing
values in the time series without the need for interpolation (Taylor and
Letham 2017). The ARIMA model on the other hand requires large outliers
to be removed and handles missing values by interpolation. If Prophet
produces forecasts that are as good as the ARIMA model when forecasting
BTC/ZAR, it can be compared to more complex forecasting methods that are
less tractable such as the hybrid models and machine learning techniques
seen earlier.

\section{\texorpdfstring{Methodology
\label{Methodology}}{Methodology }}\label{methodology}

In this paper, the out-of-sample forecasts of ARIMA and Prophet are
compared over varying forecast horizons. The data is first split into a
training set and test set. The training set starts on the 24th January
2016 and ends on the 16th January 2017, while the test set starts on the
17th January 2017 and ends on the 17th July 2017. The Box-Jenkins
methodology is then used to select the correct ARIMA model while
Prophet’s semi-automatic procedure is used for model selection. The
selected ARIMA and Prophet model are then fitted to the training set and
rolling window forecasts are made 1 day, 30 days and 90 days ahead. This
allows us to examine the forecast horizon effect. The Model Confidence
Set and Mincer-Zarnowitz test is then used to evaluate the forecasts
produced by ARIMA and Prophet against the test set. Since there is no
consensus on which accuracy statistic best measures the performance of
forecasting techniques, the most popular criteria, the Root Mean Squared
Error (RMSE) is employed. The results obtained from the ARIMA model will
be used as a benchmark for comparison. This process is then repeated
based on data which has outliers and missing values present.

\subsection{The Box-Jenkins
Methodology}\label{the-box-jenkins-methodology}

Box, Jenkins, and Reinsel (1970) proposed a set of guidelines that can
be followed when selecting ARIMA models. It consists of a four-stage
iterative process in which:

\begin{enumerate}
\def\labelenumi{\arabic{enumi}.}
\tightlist
\item
  The process is either transformed or differenced to de-trend and
  stabilise the variance of the data.
\item
  The autocorrelation and partial autocorrelation plots are used to
  determine the order of p and q.
\item
  The parameters of the model are then estimated.
\item
  A diagnostic check is performed to ensure that the residuals are a
  white noise process.
\end{enumerate}

If the residuals are not white noise, steps 2-4 are repeated until a
satisfactory model is identified. On the contrary, if the diagnostic
check reveals that the residuals are random, the developed model will be
the final model used for forecasting.

\subsection{\texorpdfstring{Semi-Automatic Selection
\label{Semi-Automatic Selection}}{Semi-Automatic Selection }}\label{semi-automatic-selection}

When a large number of forecasts are produced, manually identifying
problematic forecasts becomes a time consuming and difficult task.
Prophet provides a semi-automated forecast evaluation system that
selects the best model which fits the data. When there are large
forecast errors, the forecasts are flagged so that the analyst can
explore the cause of the errors, identify and remove potential outliers
and either adjust the model or choose a more appropriate model (Taylor
and Letham 2017). This keeps the analyst-in-the-loop. Prophet has the
following default settings:

\begin{itemize}
\tightlist
\item
  The trend is set to be linear.
\item
  The width of the uncertainty intervals is set to 80\%.
\item
  Weekly and yearly seasonality are automatically detected and included
  in the model if present
\item
  The smoothing parameter for holidays and seasonality is set at 10
  while the smoothing parameter for the trend is set to 0.05.
\item
  The number of potential changepoints is set to 25.
\end{itemize}

This paper makes use of Prophet’s default settings to fit the data,
however weekly seasonality is included in the model. This allows us to
account for our prior knowledge about the closing prices of Bitcoin
which tend to be lower around the weekend however higher mid-week due to
fluctuations in trading activity. Furthermore, a linear trend is
appropriate since Bitcoin does not have an upper limit on its closing
prices.

\subsection{Rolling Window Forecasts}\label{rolling-window-forecasts}

Rolling window forecasts are useful in evaluating the robustness of a
forecasting method. In a rolling window forecast, the forecasts are made
h-steps at a time and the actual observation rather than the predicted
value is used for the next prediction in the forecast horizon. In this
way, a poor forecast will not have negative consequences on future
forecasts to be made since the observed values will be used to correct
itself for the remaining forecasts (Zhang and Hu 1998). In this paper,
the ARIMA and Prophet model is first fitted to the training set and the
returns are forecasted h-days ahead for h = 1, 30, 90. The size of the
training set is then increased by one observation and the models are
refitted. The next h-day ahead return is forecasted and this process is
repeated until all forecasts have been made into the test set. The h-day
ahead forecasts are then compared to the h-day ahead observed values in
the test set and the forecast errors are calculated to determine the
RMSE.

\subsection{Model Confidence Set}\label{model-confidence-set}

The Model Confidence Set (MCS) for a set of models is analogous to a
confidence interval for a model parameter. The MCS procedure developed
by Hansen, Lunde, and Nason (2011) is a sequential procedure that starts
with an initial set of models \(M_0\) of size m. The worst model is then
eliminated at each step until the null hypothesis of equal predictive
ability (EPA) fails to be rejected for all the models belonging to the
set at a given confidence level \(1 - \alpha\). This smaller set of
models \(M^*\) is the superior set of models (SSM) with the best case
occurring when the SSM consists of a single best model.

The null hypothesis for EPA for a set of models \(M\) is given by:
\[H_0: \mu_i,_j = 0 \;\;\; \forall \;\; i,j = 1,...,m\]

where \(\mu_i,_j = E[d_i,_j,_t]\) represents the expected loss
differential, \(d_i,_j,_t = l_i,_t - l_j,_t\) represents the loss
differential between models \(i\) and \(j\) at time t, and \(l_i,_t\)
and \(l_j,_t\) are the loss functions associated with model \(i\) and
\(j\) at time t respectively. This paper makes use of the squared
forecast error as a loss function, however any arbitrary loss function
that satisfies the weak stationarity conditions described in Hansen,
Lunde, and Nason (2011) may be used.\\
The null hypothesis is tested by constructing the following test
statistics:

\[t_i,_j = \frac{\bar{d_i,_j}}{\sqrt{\hat{var[\bar{d_i,_j}]}}}\]
\vspace{1pt} \[ T_R = \max_{i,j\in M} \;\; |t_i,_j|\] where
\(\bar{d_i,_j} = m^{-1}\sum_{t=1}^{m}{d_i,_j,_t }\) measures the sample
average loss differential between model \(i\) and \(j\) and
\(\hat{var[\bar{d_i,_j}]}\) is estimated through the bootstrapping
procedure described in Hansen, Lunde, and Nason (2011).

This paper uses a block-bootstrap procedure of 5000 resamples. The block
length \(p\) is chosen by fitting an AR(p) model to the \(d_i,_j,_t\)
terms and determining the maximum number of significant AR parameters
obtained. While \(t_i,_j\) has a t-distribution, \(T_R\) has a
non-standard asymptotic distribution under the null hypothesis hence is
estimated by implementing a bootstrapping procedure similar to that used
to estimate \({var[\bar{d_i,_j}]}\).

The worst model is then eliminated according to an elimination rule that
is coherent with the calculated t-statistic and is defined as:

\[e_R = \arg max_{i}\bigg(\sup_{j\in M}\frac{\bar{d_i,_j}}{\sqrt{\hat{var[\bar{d_i,_j}]}}}\bigg)\]
\#\# Mincer-Zarnowitz Approach to Forecast Evaluation The
Mincer-Zarnowitz approach to evaluating forecast accuracy is commonly
used and can be useful when comparing the forecasts produced by Prophet
and the ARIMA model. Mincer and Zarnowitz (1969) proposed an absolute
measure to evaluate forecast accuracy that considers the distance
between the actual and predicted values. To analyse the absolute errors
produced by the forecasts, the observed values \(r_t\) are regressed
against the predicted values \(\hat{r_t}\), i.e.
\[r_t = \alpha + \beta\cdot\hat{r_t} + e_t\] where \(\alpha\) represents
the mean distance between the observed and predicted values while
\(\beta\) represents the correlation between the forecasted errors and
predicted values. When \(\alpha = 0\) it implies that the forecasts are
unbiased and do not systematically overestimate or underestimate the
data. When \(\beta = 1\) it implies that the forecasts are efficient and
uncorrelated with the forecasted errors. A joint hypothesis test of
\(H_0: \alpha = 0 \cup \beta = 1\) is performed to check the efficiency
and bias of the forecasts and the model which produces the best results
generates superior forecasts.

\subsection{Evaluation of forecasts based on data with missing values
and
outliers}\label{evaluation-of-forecasts-based-on-data-with-missing-values-and-outliers}

The original dataset is modified in order to examine how robust the
ARIMA and Prophet model are when faced with missing values and outliers.
20 values are randomly removed from the training set and 5 outliers are
randomly inserted in place of existing values. Both models are then
fitted to this “dirty” dataset and the forecasts are evaluated in
the same way as the ``clean'' dataset.

\section{\texorpdfstring{Results
\label{Results}}{Results }}\label{results}

\subsection{Evaluation of clean dataset over varying forecast
horizons}\label{evaluation-of-clean-dataset-over-varying-forecast-horizons}

By following the Box-Jenkins methodology, an ARIMA(0,0,1) model was
found to be the best fit for the clean training set. Making use of a
mean model would result in a returns forecast of zero in a rolling
window context. This is due to the white noise behaviour of log returns.
Hence a decision was made to employ the ARIMA(1,0,0) model instead. The
ARIMA(1,0,0) model has a marginally higher AIC and an analysis of the
p-value indicated that the AR parameter is statistically significant in
the explanation of the movement of BTC/ZAR. The Prophet model fitted to
the clean training set included a linear trend and weekly seasonality
with all other parameters set to the default values as described in
section \ref{Semi-Automatic Selection}. The models chosen manually are
consistent with the models selected by the automatic-selection procedure
for ARIMA and Prophet.

Table \ref{tab1} displays the MZ results obtained by regressing the test
set returns against the forecasted returns, the RMSE, and the models
that are included in the MCS. It is evident that the ARIMA model ranks
ahead of the Prophet model over all forecast horizons. The MCS p-values
indicates that both models lie within the 90\% confidence interval at
the 1 day and 3 month forecast horizon, while the Prophet model is
eliminated from the MCS at the 1 month forecast horizon. This suggests
that the ARIMA and Prophet model have equal predictive accuracy at the 1
day and 3 month forecast horizon, and that the ARIMA model produces
superior forecasts at the 1 month forecast horizon. Furthermore, the MCS
p-values indicates that while ARIMA model lies within the 90\%
confidence interval with certainty across all forecast horizons, the
Prophet model lies within the 90\% confidence interval 52.44\% of the
time at the 1 day forecast horizon and 45.78\% of the time at the 3
month forecast horizon.

\begin{table}[H]
\centering
\caption{MCS, MZ-test and RMSE results based on clean data \label{tab1}} 
\scalebox{0.9}{
\begin{tabular}{cccccccc}
  \hline
Forecast & Model & Model & MCS & $\alpha$ & $\beta$ & MZ-test & RMSE \\ 
 Horizon &  & Rank & p-values &  &  & p-values &  \\ 
   \hline
1 & ARIMA & 1 & 1 & 0.009546 & -2.005624 & 0.02545 & 0.04406 \\ 
   & Prophet & 2 & 0.5244 & 0.013702 & -1.709744 & 0.007152 & 0.04425 \\ 
   \hline
30 & ARIMA & 1 & 1 & 0.031320 & -12.736757 & 0.06136 & 0.04459 \\ 
   & Prophet & 2 & 0.023 * & 0.017169 & -2.621914 & 0.0003773 & 0.04556 \\ 
   \hline
90 & ARIMA & 1 & 1 & 0.006012 & -0.938687 & 0.8430658 & 0.04231 \\ 
   & Prophet & 2 & 0.4578 & 0.006953 & -0.615435 & 0.4086752 & 0.04259 \\ 
   \hline
\end{tabular}
}
\end{table}

The intercept estimate represented by \(\alpha\) is small and positive
for ARIMA and Prophet across all forecast horizons, with the smallest
estimates obtained at the longest forecast horizon. This suggests that
both models systematically underestimates the returns, but to a smaller
degree at long forecast horizons. The intercept estimate for ARIMA is
closer to zero compared to Prophet at the 1 day and 3 month forecast
horizon, indicating that the mean forecasts yielded by ARIMA is closer
to the observed mean than Prophet. The slope estimate represented by
\(\beta\) is negative for both models across all forecast horizons,
indicating that the observed returns are actually the opposite of the
suggested forecast and that both models fail to capture the explosive
behaviour of BTC/ZAR.

The MZ p-values at the 1 day and 1 month forecast horizon show that the
joint hypothesis of a unity slope and zero intercept is rejected at the
10\% significance level for both models. This suggests that the
forecasts generated by ARIMA and Prophet are biased and/or inefficient.
Both models only produce unbiased and/or efficient forecasts at the
longest forecast horizon, with the ARIMA model yielding a considerably
larger p-value compared to Prophet. The RMSE produced by the ARIMA model
is marginally smaller than Prophet across all forecasts horizons, with
the smallest RMSE calculated at the longest forecast horizon. These
results agree with the results obtained from the MCS and the MZ-test.

\subsection{Evaluation of data with missing values and outliers
present}\label{evaluation-of-data-with-missing-values-and-outliers-present}

By following the Box-Jenkins methodology, an ARIMA(0,0,3) model was
found to be the best fit for the dirty training set. A decision was made
to employ the ARIMA(3,0,0) model instead. The motivation behind this
choice is the same as that provided in the case of the clean data. The
ARIMA(3,0,0) model has a marginally higher AIC and an analysis of the
p-value indicated that the AR parameters are statistically significant
in the explanation of the movement of BTC/ZAR. The Prophet model fitted
to the clean training set included a linear trend and weekly seasonality
with all other parameters set to the default values as described in
section \ref{Semi-Automatic Selection}. The models chosen manually are
consistent with the models selected by the automatic-selection procedure
for ARIMA and Prophet.

Table \ref{tab2} displays the MZ results obtained by regressing the test
set returns against the forecasted returns, the RMSE and the models that
are included in the MCS. We observe that the presence of outliers and
missing values in the data has no effect on the rank of the ARIMA and
Prophet model. Contrary to the results from the clean data, the Prophet
model lies within the 90\% confidence interval at the 1 day forecast
horizon only. This suggests that the ARIMA and Prophet model only have
equal predictive accuracy at the shortest forecast horizon. The MCS
p-values indicates that while the ARIMA model always lies within the
90\% confidence interval with certainty, the Prophet model only lies
within the confidence interval at the 1 day forecast horizon 41.3\% of
the time. This differs to the results from the clean data in which the
Prophet model lies within the 90\% confidence interval 52.44\% of the
time at the 1 day forecast horizon.

\begin{table}[H]
\centering
\caption{MCS, MZ-test and RMSE results based on dirty data \label{tab2}} 
\scalebox{0.9}{
\begin{tabular}{cccccccc}
  \hline
Forecast & Model & Model & MCS & $\alpha$ & $\beta$ & MZ-test & RMSE \\ 
 Horizon &  & Rank & p-values &  &  & p-values &  \\ 
   \hline
1 & ARIMA & 1 & 1 & 0.011895 & -0.8832 & 0.00003182 & 0.045530 \\ 
   & Prophet & 2 & 0.413 & 0.014802 & -0.6761 & 0.00000532 & 0.046210 \\ 
   \hline
30 & ARIMA & 1 & 1 & 0.114100 & -12.9853 & 0.1075 & 0.044758 \\ 
   & Prophet & 2 & 0.0016* & 0.014400 & -0.6516 & 0.00000148 & 0.048180 \\ 
   \hline
90 & ARIMA & 1 & 1 & -0.226700 & 27.0635 & 0.01294 & 0.042302 \\ 
   & Prophet & 2 & 0.0336* & -0.009291 & 0.6438 & 0.000764 & 0.045061 \\ 
   \hline
\end{tabular}
}
\end{table}

We notice that a positive intercept and negative slope is still
estimated at the 1 day and 1 month forecast horizon when the data is
dirty. The intercept estimates are larger than the case of the clean
data, suggesting that the biasness of the forecasts yielded by both
models increases when there are outliers and missing values present in
the data. Contrary to the results from the clean data, Prophet’s
intercept estimate lies closer to zero at the 1 month and 3 month
forecast horizon. This indicates that the mean forecasted returns
yielded by Prophet is closer to the mean observed returns when outliers
and missing values are present in the data. The negative intercept and
positive slope estimated for the 3 month horizon contrasts with the
positive intercept and negative slope estimated in the case of the clean
data. This shows that although the forecasts produced by both models are
systematically overestimated at long forecast horizons when the data is
dirty, they are no longer the opposite of the observed returns.

The p-values obtained from the F-test for the joint hypothesis of a
unity slope and zero intercept is approximately zero for ARIMA and
Prophet across all forecast horizons, with the p-value for Prophet lying
closer to zero than ARIMA. The null hypothesis is thus rejected at the
10\% significance level, indicating that the forecasts generated by
Prophet and ARIMA are always biased and/or inefficient. This contrasts
with the clean data which generated unbiased and/or efficient forecasts
at the longest forecast horizon. The RMSE produced by both models agrees
with the MZ-test and MCS results, with ARIMA producing a marginally
smaller RMSE than Prophet across all forecast horizons, and the RMSE
being larger than the case of the clean data.

\section{\texorpdfstring{Discussion and Conclusions
\label{Discussion and Conclusions}}{Discussion and Conclusions }}\label{discussion-and-conclusions}

The results show a clear difference in the predictive accuracy of the
ARIMA and Prophet model in addition to the forecast horizon effect. The
ARIMA model ranks ahead of Prophet when forecasting the returns of
BTC/ZAR across all forecast horizons. These results hold true
irrespective of whether there are missing values and outliers present in
the data. Nevertheless, it is reassuring to see that both models have
equal predictive accuracy when forecasting at short and long forecast
horizons when the data is clean and at short forecast horizons when the
data is dirty. Furthermore, the Mincer-Zarnowitz p-values for both
models are significant at the same forecast horizons, suggesting that
the efficiency and biasness of the return forecasts yielded by ARIMA and
Prophet are synchronized at each forecast horizon. Hence, if the analyst
is engaged in short-term or long-term trading, they could choose to
employ the Prophet model instead of the ARIMA model. This would allow
them to produce return forecasts that are as good as the ARIMA model,
without having any expert knowledge about model building and selection.
It would also provide them with the flexibility to customise the model
to suit their needs and incorporate their domain knowledge about BTC/ZAR
through the model’s intuitively adjustable parameters. If the analyst
is interested in generating unbiased and/or efficient forecasts with the
least amount of forecast error, they could utilise either model, however
should only forecast BTC/ZAR returns over long forecast horizons with
data that do not include any outliers or missing values.

The results from the dirty data show that the ARIMA and Prophet model
are sensitive to the presence of outliers and missing values, with both
models producing unbiased and/or inefficient forecasts and higher RMSEs
across all forecast horizons. This is disappointing as Prophet is said
to be robust when faced with outliers and can handle missing values
without the need for interpolation unlike the ARIMA model. Since both
models have equal predictive accuracy at the shortest forecast horizon,
an analyst engaged in short-term trading who has no knowledge on how to
pre-process BTC/ZAR data before forecasting, could apply the Prophet
model as it would provide them with flexibility and ease of
implementation without sacrificing any predictive accuracy.

Although the Prophet model was eliminated from the Model Confidence Set
at other forecast horizons, these results might change as more data and
information about events which might impact the price of BTC/ZAR becomes
available and is incorporated into the Prophet model. Furthermore, the
RMSE shows that the difference between the forecast errors of ARIMA and
Prophet is marginal, even in the case where both models don’t have
equal predictive accuracy. Hence, if the analyst is willing to sacrifice
a small amount of predictive accuracy in return for the tractability and
ease of implementation which Prophet provides, then they might favour
Prophet over the ARIMA model.

The attractive features of Prophet were not fully utilised in this paper
due to the stationarity of log returns. Different results may emerge
when forecasting the direction of returns or closing prices.
Furthermore, the reader should exercise caution when generalising these
results to other exchange rate data or to exchange rates which involve
fiat currencies as results may differ. As a point of further research,
more complex techniques such as Neural Networks could be used as a
benchmark for comparison when evaluating the forecasts produced by
Prophet.

\newpage

\section*{References}\label{references}
\addcontentsline{toc}{section}{References}

\hypertarget{refs}{}
\hypertarget{ref-abu1996}{}
Abu-Mostafa, Yaser S, and Amir F Atiya. 1996. ``Introduction to
Financial Forecasting.'' \emph{Applied Intelligence} 6 (3). Springer:
205--13.

\hypertarget{ref-anastasakis2009}{}
Anastasakis, Leonidas, and Neil Mort. 2009. ``Exchange Rate Forecasting
Using a Combined Parametric and Nonparametric Self-Organising Modelling
Approach.'' \emph{Expert Systems with Applications} 36 (10). Elsevier:
12001--11.

\hypertarget{ref-box1970}{}
Box, George EP, Gwilym M Jenkins, and G Reinsel. 1970. ``Forecasting and
Control.'' \emph{Time Series Analysis} 3: 75.

\hypertarget{ref-castillo2002}{}
Castillo, Oscar, and Patricia Melin. 2002. ``Hybrid Intelligent Systems
for Time Series Prediction Using Neural Networks, Fuzzy Logic, and
Fractal Theory.'' \emph{IEEE Transactions on Neural Networks} 13 (6).
IEEE: 1395--1408.

\hypertarget{ref-cheung1993}{}
Cheung, Yin-Wong. 1993. ``Long Memory in Foreign-Exchange Rates.''
\emph{Journal of Business \& Economic Statistics} 11 (1). Taylor \&
Francis: 93--101.

\hypertarget{ref-etuk2013}{}
Etuk, Ette Harrison, Dagogo SA Wokoma, and Imoh Udo Moffat. 2013.
``Additive Sarima Modelling of Monthly Nigerian Naira-Cfa Franc Exchange
Rates.'' \emph{European Journal of Statistics and Probability} 1 (1):
1--12.

\hypertarget{ref-fahimifard2009}{}
Fahimifard, SM, Masuod Homayounifar, M Sabouhi, and AR Moghaddamnia.
2009. ``Comparison of Anfis, Ann, Garch and Arima Techniques to Exchange
Rate Forecasting.'' \emph{Journal of Applied Sciences} 9 (20): 3641--51.

\hypertarget{ref-fantazzini2016}{}
Fantazzini, Dean, Erik Nigmatullin, Vera Sukhanovskaya, and Sergey
Ivliev. 2016. ``Everything You Always Wanted to Know About Bitcoin
Modelling but Were Afraid to Ask.''

\hypertarget{ref-hansen2011}{}
Hansen, Peter R, Asger Lunde, and James M Nason. 2011. ``The Model
Confidence Set.'' \emph{Econometrica} 79 (2). Wiley Online Library:
453--97.

\hypertarget{ref-hyndman2014}{}
Hyndman, Rob J, and George Athanasopoulos. 2014. \emph{Forecasting:
Principles and Practice}. OTexts.

\hypertarget{ref-khashei2011}{}
Khashei, Mehdi, and Hehdi Bijari. 2011. ``Exchange Rate Forecasting
Better with Hybrid Artificial Neural Networks Models.'' \emph{Journal of
Mathematical and Computational Science} 1 (1). Science \& Knowledge
Publishing Corporation Limited (SCIK): 103.

\hypertarget{ref-khashei2009}{}
Khashei, Mehdi, Mehdi Bijari, and Gholam Ali Raissi Ardali. 2009.
``Improvement of Auto-Regressive Integrated Moving Average Models Using
Fuzzy Logic and Artificial Neural Networks (Anns).''
\emph{Neurocomputing} 72 (4). Elsevier: 956--67.

\hypertarget{ref-lin2012}{}
Lin, Chiun-Sin, Sheng-Hsiung Chiu, and Tzu-Yu Lin. 2012. ``Empirical
Mode Decomposition--based Least Squares Support Vector Regression for
Foreign Exchange Rate Forecasting.'' \emph{Economic Modelling} 29 (6).
Elsevier: 2583--90.

\hypertarget{ref-mincer1969}{}
Mincer, Jacob A, and Victor Zarnowitz. 1969. ``The Evaluation of
Economic Forecasts.'' In \emph{Economic Forecasts and Expectations:
Analysis of Forecasting Behavior and Performance}, 3--46. NBER.

\hypertarget{ref-nwankwo2014}{}
Nwankwo, Steve C. 2014. ``Autoregressive Integrated Moving Average
(Arima) Model for Exchange Rate (Naira to Dollar).'' \emph{Academic
Journal of Interdisciplinary Studies} 3 (4): 429.

\hypertarget{ref-santos2007}{}
Santos, André Alves Portela, Newton Carneiro Affonso da Costa, and
Leandro dos Santos Coelho. 2007. ``Computational Intelligence Approaches
and Linear Models in Case Studies of Forecasting Exchange Rates.''
\emph{Expert Systems with Applications} 33 (4). Elsevier: 816--23.

\hypertarget{ref-taylor2017}{}
Taylor, Sean J, and Benjamin Letham. 2017. ``Forecasting at Scale.''

\hypertarget{ref-tseng2001}{}
Tseng, Fang-Mei, Gwo-Hshiung Tzeng, Hsiao-Cheng Yu, and Benjamin JC
Yuan. 2001. ``Fuzzy Arima Model for Forecasting the Foreign Exchange
Market.'' \emph{Fuzzy Sets and Systems} 118 (1). Elsevier: 9--19.

\hypertarget{ref-zhang1998}{}
Zhang, Gioqinang, and Michael Y Hu. 1998. ``Neural Network Forecasting
of the British Pound/Us Dollar Exchange Rate.'' \emph{Omega} 26 (4).
Elsevier: 495--506.

\newpage
\renewcommand{\baselinestretch}{1}
\nocite{*}
\bibliography{}

\end{document}
