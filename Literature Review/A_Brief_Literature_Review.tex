\documentclass[12pt,preprint, authoryear]{elsarticle}

\usepackage{lmodern}
%%%% My spacing
\usepackage{setspace}
\setstretch{1.5}
\DeclareMathSizes{12}{14}{10}{10}

% Wrap around which gives all figures included the [H] command, or places it "here". This can be tedious to code in Rmarkdown.
\usepackage{float}
\let\origfigure\figure
\let\endorigfigure\endfigure
\renewenvironment{figure}[1][2] {
    \expandafter\origfigure\expandafter[H]
} {
    \endorigfigure
}

\let\origtable\table
\let\endorigtable\endtable
\renewenvironment{table}[1][2] {
    \expandafter\origtable\expandafter[H]
} {
    \endorigtable
}


\usepackage{ifxetex,ifluatex}
\usepackage{fixltx2e} % provides \textsubscript
\ifnum 0\ifxetex 1\fi\ifluatex 1\fi=0 % if pdftex
  \usepackage[T1]{fontenc}
  \usepackage[utf8]{inputenc}
\else % if luatex or xelatex
  \ifxetex
    \usepackage{mathspec}
    \usepackage{xltxtra,xunicode}
  \else
    \usepackage{fontspec}
  \fi
  \defaultfontfeatures{Mapping=tex-text,Scale=MatchLowercase}
  \newcommand{\euro}{€}
\fi

\usepackage{amssymb, amsmath, amsthm, amsfonts}

\usepackage[round]{natbib}
\bibliographystyle{natbib}
\def\bibsection{\section*{References}} %%% Make "References" appear before bibliography
\usepackage{longtable}
\usepackage[margin=2cm,bottom=4cm,top=2.5cm, includefoot]{geometry}
\usepackage{fancyhdr}
\usepackage[bottom, hang, flushmargin]{footmisc}
\usepackage{graphicx}
\numberwithin{equation}{section}
\numberwithin{figure}{section}
\numberwithin{table}{section}
\setlength{\parindent}{0cm}
\setlength{\parskip}{1.3ex plus 0.5ex minus 0.3ex}
\usepackage{textcomp}
\renewcommand{\headrulewidth}{0.2pt}
\renewcommand{\footrulewidth}{0.3pt}

\usepackage{array}
\newcolumntype{x}[1]{>{\centering\arraybackslash\hspace{0pt}}p{#1}}

%%%%  Remove the "preprint submitted to" part. Don't worry about this either, it just looks better without it:
\makeatletter
\def\ps@pprintTitle{%
  \let\@oddhead\@empty
  \let\@evenhead\@empty
  \let\@oddfoot\@empty
  \let\@evenfoot\@oddfoot
}
\makeatother

 \def\tightlist{} % This allows for subbullets!

\usepackage{hyperref}
\hypersetup{breaklinks=true,
            bookmarks=true,
            colorlinks=true,
            citecolor=blue,
            urlcolor=blue,
            linkcolor=blue,
            pdfborder={0 0 0}}

\urlstyle{same}  % don't use monospace font for urls
\setlength{\parindent}{0pt}
\setlength{\parskip}{6pt plus 2pt minus 1pt}
\setlength{\emergencystretch}{3em}  % prevent overfull lines
\setcounter{secnumdepth}{5}

%%% Use protect on footnotes to avoid problems with footnotes in titles
\let\rmarkdownfootnote\footnote%
\def\footnote{\protect\rmarkdownfootnote}
\IfFileExists{upquote.sty}{\usepackage{upquote}}{}

%%% Include extra packages specified by user
% Insert custom packages here as follows
% \usepackage{tikz}

\begin{document}

\begin{frontmatter}  %

\title{Literature Review}

\author[Add1]{Fiona Ganie}
\ead{GNXFIO001@myuct.ac.za}





\address[Add1]{University of Cape Town}



\vspace{1cm}

\begin{keyword}
\footnotesize{
 \\ \vspace{0.3cm}
\textit{JEL classification} 
}
\end{keyword}
\vspace{0.5cm}
\end{frontmatter}



%________________________
% Header and Footers
%%%%%%%%%%%%%%%%%%%%%%%%%%%%%%%%%
\pagestyle{fancy}
\chead{}
\rhead{}
\lfoot{}
\rfoot{\footnotesize Page \thepage\\}
\lhead{}
%\rfoot{\footnotesize Page \thepage\ } % "e.g. Page 2"
\cfoot{}

%\setlength\headheight{30pt}
%%%%%%%%%%%%%%%%%%%%%%%%%%%%%%%%%
%________________________

\headsep 35pt % So that header does not go over title




\section{\texorpdfstring{Introduction
\label{Introduction}}{Introduction }}\label{introduction}

The ability to forecast exchange rates is of great importance to policy
makers, businesses and speculators. Policy makers are interested in
forecasting exchange rates to assess how its movement will impact
inflation, and how monetary policy will need to be adjusted in the
future to keep inflation within the inflation target (Fahimifard et al.
\protect\hyperlink{ref-fahimifard2009}{2009}). Businesses need to assess
how future movements in exchange rates will affect the cost of importing
goods and the demand for goods exported, while speculators aim to
identify opportunities in price movements which can be exploited to make
a profit.

Producing accurate forecasts of exchange rates has proven to be a
difficult task due to the noise of the time series (Abu-Mostafa and
Atiya \protect\hyperlink{ref-abu1996}{1996}). Some economists argue that
the exchange rate market is very efficient and future movements in price
are unpredictable (Abu-Mostafa and Atiya
\protect\hyperlink{ref-abu1996}{1996}). This belief in the efficient
market hypothesis has led many researchers to model exchange rates using
a Random Walk (RW) model (Abu-Mostafa and Atiya
\protect\hyperlink{ref-abu1996}{1996}). Although the RW model has
produced superior forecasts, some researchers believe that future prices
depend on present and past events, and other techniques such as the
autoregressive integrated moving average (ARIMA) model has been applied
to forecast exchange rates (Abu-Mostafa and Atiya
\protect\hyperlink{ref-abu1996}{1996}).

The ARIMA model is one of the most commonly used methods in financial
forecasting (Fahimifard et al.
\protect\hyperlink{ref-fahimifard2009}{2009}). It is known for its
attractive quality of being a rich class of processes, making it
possible to find a model that adequately fits the data (Fahimifard et
al. \protect\hyperlink{ref-fahimifard2009}{2009}). Some research has
shown that the nature of exchange rate data is non-linear, and this
non-linearity may not be captured adequately by the ARIMA model (Zhang
and Hu \protect\hyperlink{ref-zhang1998}{1998}). Linear parametric
forecasting techniques such as the generalised autoregressive
conditional heteroskedasticity (GARCH) model has been proposed as a
solution to this problem (Zhang and Hu
\protect\hyperlink{ref-zhang1998}{1998}). Although the GARCH model takes
care of the non-linearity in the data, the pre-specification of the
model may prevent it from capturing all the different non-linearities
which may be present in the data.

With the evolution of computational power, other non-linear
non-parametric techniques such as Artificial Neural Networks (ANN),
Support Vector Regression (SVR), Fuzzy Logistic Systems and hybrid
versions of the ARIMA model have been proposed. The ARIMA model has been
used as the standard benchmark model against which these other methods
have been compared. Although they have produced significantly better
results than the ARIMA model, building these models is a complex task.
The ARIMA model is more tractable, less computationally expensive, and
has been used as the building blocks for more advanced models such as
the autoregressive fractionally integrated (ARFIMA), Fuzzy ARIMA, and
seasonal ARIMA (SARIMA) model which have been used for exchange rate
forecasting.

Facebook recently open sourced their forecasting model Prophet. Prophet
is said to be preferable over the ARIMA model as it has the ability to
forecast at scale, is easily interpretable and has parameters that can
be intuitively adjusted (Taylor and Letham
\protect\hyperlink{ref-taylor2017}{2017}). This paper aims to compare
the forecasts produced by the traditional ARIMA model and Prophet
through an evaluation of Bitcoin/ZAR, using a Mincer-Zarnowitz approach
of measuring forecast accuracy. Section 2 will consider the origins of
the ARIMA model and how it has performed and evolved over time by
exploring empirical studies where it has been used for forecasting
exchange rates. Section 3 will also consider other techniques that have
been used to forecast exchange rates and how they have been compared to
the ARIMA model. Section 4 will look at hybrid models that were inspired
by the ARIMA model, while section 5 looks at forecasting with Prophet.
Finally, section 6 will look at the Mincer-Zarnowitz approach to
evaluating forecasts.

\section{Forecasting Exchange Rates with the ARIMA
Model}\label{forecasting-exchange-rates-with-the-arima-model}

\subsection{Birth of ARIMA}\label{birth-of-arima}

ARIMA models became highly popular since its introduction by Box and
Jenkins, when it was shown that they could outperform complex
econometric models in a variety of situations (Hibon and Makridakis
\protect\hyperlink{ref-spyros1997}{1997}). The ARIMA model expresses the
process \{\(y_t\)\} as a function of the weighted average of past values
of the process and lagged values of the residuals. The weighted average
of the past p values of the process represents an autoregressive (AR)
process of order p.~It feeds back past values of the process into the
current value, inducing correlation between all lags of the process. The
weighted average of the q lagged residuals represents a moving average
(MA) process of order q. The purpose of mixing the MA process with the
AR process is to reduce the large number of past values required by AR
processes and to control for the autocorrelation which it creates
between lagged values of the process. The combination of the AR(p) and
MA(q) process results in a more parsimonious model, and forms a
stationary autoregressive moving average (ARMA(p,q)) process defined as:

\[ 
    y_t = c + \sum_{i = 1}^{p} \theta_{i} y_{t-i} + \sum_{i = 1}^{q} \phi_{i} e_{t-i}+ e_t \label{eq1} \\ \notag
\] where c is a constant and \{\(e_t\)\} is a white noise process with
zero mean and variance \(\sigma^2\).

The ARIMA(p,d,q) model generalises the ARMA model in that it includes
both stationary and non-stationary processes. The parameter d is the
degree of differencing required to render the process stationary. If d
is equal to zero the process is stationary and equivalent to an ARMA
model, and if d is strictly positive the process requires differencing
to make it stationary. The ARIMA model can be defined succinctly using
the backward shift operator B, which shifts the process back by one unit
of time, and is defined as \(By_t = y_{t-1}\). The ARIMA model has the
form:

\[ 
    (1-\sum_{i = 1}^{p} \theta_{i} B^i)(1-B)^dy_t = c + (1 + \sum_{i = 1}^{q} \phi_{i} B^i)e_t \label{eq2} \\ \notag 
\] where c is a constant and \{\(e_t\)\} is a white noise process with
zero mean and variance \(\sigma^2\).

\subsection{The Box-Jenkins Methodology and Automatic
Selection}\label{the-box-jenkins-methodology-and-automatic-selection}

Box and Jenkins proposed a set of guidelines that can be followed when
selecting ARIMA models. This systematic procedure to designing ARIMA
models has made them highly popular (Hibon and Makridakis
\protect\hyperlink{ref-spyros1997}{1997}). It consists of a four-stage
iterative process in which: 1) the process is either transformed or
differenced to detrend and stabilise the variance of the data, 2) the
autocorrelation and partial autocorrelation plots are used to determine
the order of p and q, 3) the parameters of the model are estimated, 4) a
diagnostic check is performed to ensure that the residuals are a white
noise process. If the residuals are not white noise steps 2-4 are
repeated until a satisfactory model is identified. On the contrary, if
the diagnostic check shows that the residuals are random then the
developed model is the final model used for forecasting.

Due to the large number of forecasts made, it is useful to have an
automatic procedure which is able to select the appropriate ARIMA model
to fit the data. The auto.arima function in R is able to automatically
choose the order of the parameters p, q and d. The order of differencing
is determined by using the KPSS test (Ruppert and Matteson
\protect\hyperlink{ref-ruppert2015}{2015}). The KPSS test checks the
null hypothesis of stationarity and sets d to zero if the null
hypothesis is accepted, otherwise it iteratively increases d by 1 and
tests the null hypothesis until it is accepted (Ruppert and Matteson
\protect\hyperlink{ref-ruppert2015}{2015}). Once the order of
differencing has been determined, the order of p and q are chosen based
on Akaike's Information Criterion (AIC) or the Bayesian Information
Criterion (BIC) (Ruppert and Matteson
\protect\hyperlink{ref-ruppert2015}{2015}).

\subsection{Empirical Applications}\label{empirical-applications}

ARIMA models have been commonly used in financial forecasting and are
popular for observing stock prices and exchange rates due to its power
and statistical properties (C.-S. Lin, Chiu, and Lin
\protect\hyperlink{ref-lin2012}{2012}). They have been used as a
benchmark to compare new forecasting techniques which have emerged over
time. Kamruzzaman and Sarker
(\protect\hyperlink{ref-kamruzzaman2003}{2003}) applied an ARIMA model
to forecast the exchange rate between the Australian dollar and six
other currencies. Its results were used to evaluate the performance of
three Neural Network models. The Normalised Mean Square Error (NMSE) and
Mean Absolute Error (MAE) were used to measure the forecast errors while
the Directional Symmetry (DS), Correct Up trend (CU) and Correct Down
trend (CD) were used to measure the accuracy of the direction of the
forecasts. Similarly, C.-S. Lin, Chiu, and Lin
(\protect\hyperlink{ref-lin2012}{2012}) employed an ARIMA model to
forecast the Taiwan New Dollar exchange rate. The aim of applying the
ARIMA model was to analyse the ability of a proposed hybrid model
created from the Squares Support Vector Regression (LSSVR) and Empirical
Mode Decomposition (EMD) model. C.-S. Lin, Chiu, and Lin
(\protect\hyperlink{ref-lin2012}{2012}) used the same indicators as
Kamruzzaman and Sarker (\protect\hyperlink{ref-kamruzzaman2003}{2003})
to evaluate the accuracy of the direction of the forecasts. To measure
the forecast errors produced they applied the mean absolute percentage
error (MAPE), root‐mean‐square error (RMSE) and mean absolute difference
(MAD). Khashei, Bijari, and Ardali
(\protect\hyperlink{ref-khashei2012}{2012}) also investigated the
ability to forecast exchange rates by considering the British pound and
US dollar, over a short and long forecast horizon. They compared the
predictive capability of their proposed hybrid model consisting of an
ARIMA and a Probabilistic Neural Network (PNN), to the traditional ARIMA
model, and justified the use of the ARIMA as a benchmark, by claiming
that it is the most important linear model. The ARIMA model has also
yielded satisfactory results when predicting exchange rates. Nwankwo
(\protect\hyperlink{ref-nwankwo2014}{2014}) forecasted the Nairo/dollar
rate and diagnostic testing revealed that the ARIMA(1,0,0) model was the
best fit for the data based on the AIC.

The ARIMA model is attractive as it is tractable and produces good
forecasts in short time periods where more than 100 observations are
used (Tseng et al. \protect\hyperlink{ref-tseng2001}{2001}). Although
the ARIMA model has the advantage of ease of implementation and
flexibility, it fails to capture the non-linearity and volatility
present in exchange rate data.

\subsection{Evolution of ARIMA}\label{evolution-of-arima}

Over time, the ARIMA model has evolved to cater for a wider variety of
data that it can be applied to, and to compensate for some of its
shortcomings. The most popular versions of the ARIMA model that has been
implemented in exchange rate forecasting is the Seasonal ARIMA (SARIMA)
and Fractional ARIMA model (FARIMA).

The SARIMA model was introduced by Box and Jenkins as a generalisation
of the ARIMA model to capture the periodic behaviour of data. SARIMA
models extend the ARIMA class by including a term for seasonal
differencing. Exchange rate data that has been observed have shown
seasonality, and have been attempted to be fitted by SARIMA models
(Etuk, Wokoma, and Moffat \protect\hyperlink{ref-etuk2013}{2013}). Etuk,
Wokoma, and Moffat (\protect\hyperlink{ref-etuk2013}{2013}) modelled the
Naira/CFA Franc exchange rate which exhibited monthly seasonality using
an additive SARIMA model, to demonstrate that it can be a useful fit to
the data. Their results showed that the SARIMA model adequately
described the variation in the exchange rate series. This agrees with
the results of a study by Etuk and others
(\protect\hyperlink{ref-etuk2012}{2012}) which fitted a multiplicative
SARIMA model to the Naira/Dollar exchange rates.

The ARFIMA model is another generalisation of the ARIMA model in that
the degree required to make the data stationary can assume any real
value, and is no longer restricted to the integer domain. While the
ARIMA model displays exponentially decaying autocorrelation between
observations the further apart they are in time, the ARFIMA model has
the ability to capture the dependence between observations that are
widely spread apart in time (Cheung
\protect\hyperlink{ref-cheung1993}{1993}). This makes the ARFIMA model
more parsimonious since it can capture long memory in data as well as
short term dynamics (Cheung \protect\hyperlink{ref-cheung1993}{1993}).
Cheung (\protect\hyperlink{ref-cheung1993}{1993}) fitted the ARFIMA
model to examine five exchange rates, and found that there was strong
evidence of long memory in the exchange rate time series. Although the
ARFIMA model fitted the data well, it failed to beat the random walk in
out-of sample forecasts.

GARCH models were later developed due to the failure of ARIMA models to
capture the volatility in financial markets (Anastasakis and Mort
\protect\hyperlink{ref-anastasakis2009}{2009}). They have been widely
used to forecast the volatility of stock returns, interest rates and
exchange rates and are useful for portfolio allocation decisions as well
as option pricing (Cermak \protect\hyperlink{ref-cermak2017}{2017}).
Cermak (\protect\hyperlink{ref-cermak2017}{2017}) employed a GARCH(1,1)
model to analyse the volatility of Bitcoin in countries in which Bitcoin
is commonly traded, while Hsieh (1989) applied a GARCH model to
investigate five exchange rates. Hsieh's (1989) results showed that
although the GARCH model outperformed the random walk, some non-linear
information still remained in the residuals. In a similar study,
Fahimifard et al. (\protect\hyperlink{ref-fahimifard2009}{2009})
illustrated that the GARCH model outperformed the ARIMA, while Neural
Networks outperformed both the GARCH and ARIMA model.

\section{Other techniques used to forecast exchange
rates}\label{other-techniques-used-to-forecast-exchange-rates}

Over time, financial forecasting methods have moved away from linear
models like ARIMA and GARCH, to soft computing techniques. These more
complex techniques are non-linear and can fit complex time series more
easily (Castillo and Melin \protect\hyperlink{ref-castillo2002}{2002}).
Unlike regression analysis and the ARIMA model, soft computing
techniques do not impose structural assumptions on the model apriori
(Castillo and Melin \protect\hyperlink{ref-castillo2002}{2002}). Some of
the most commonly used artificial intelligence methods used to forecast
exchange rate data are Neural Networks and the Fuzzy Logistic Systems.

\subsection{Neural Networks}\label{neural-networks}

Artificial Neural Networks are commonly used in financial forecasting
and are more advantageous than other non-linear forecasting methods.
They can approximate any continuous function, are data driven, and can
adapt to non-stationary environments (Khashei and Bijari
\protect\hyperlink{ref-khashei2011}{2011}). Neural Networks mimic the
structure of the brain and consists of three layers: the input, hidden
and output layer, each of which contain nodes which are representative
of the neurons in our brain. The job of the nodes in the input layer is
to send signals along connections to the nodes in the hidden layer,
where the connections are representative of the synapses in our brain.
The nodes in the hidden layer uses an activation function and assigns a
weight to each connection according to the level of importance of
information. The value in the hidden layer is then sent along
connections to the output layer, where weights are again assigned to
each connection. The error is then computed in the output layer and is
backpropagated through the network. The weights are adjusted according
to the error that they contribute, allowing the ANN to be trained and
learn. This learning process is repeated until the network achieves the
desired output.

ANN's have had many successful applications in forecasting exchange
rates. In a study done by Fahimifard et al.
(\protect\hyperlink{ref-fahimifard2009}{2009}), the ANN was found as an
effective way to improve the forecasts of exchange rates. Superior
results were produced when compared to the ARIMA and GARCH model using
the RMSE, MSE and MAD as measure of performance. Similar results were
found by Zhang and Hu (\protect\hyperlink{ref-zhang1998}{1998}) however
the ANN failed to outperform the random walk over long forecast
horizons. Contrary to the findings above, Kuan and Liu
(\protect\hyperlink{ref-kuan1995}{1995}) attempted to investigate how
well feedforward and recurrent networks capture the non-linearity of
exchange rate data and obtained conflicting results with neural networks
having significantly better market timing ability or lower out-of-sample
Mean Square Prediction Error (MSPE) compared to the ARMA models for some
exchange rates only. Franses and Van Homelen
(\protect\hyperlink{ref-franses1998}{1998}) suggested that the
non-linear feature of exchange rates that are picked up by ANNs may
actually be due to neglected GARCH effects. They investigated the
application of ANNs to forecast exchange rates and found that the
presence of GARCH in the data may mistakenly lead one to believe that
the returns can be forecasted on the exchange rates themselves. Their
results show that there is no gain in producing out-of-sample forecasts
using ANN if the data is not truly non-linear.

Although neural networks have been broadly applied in financial
forecasting, the process of building neural networks is a complex task
and there is no consistent method to design them compared to the
traditional Box-Jenkins ARIMA model. Unlike ARIMA models, the
performance of ANNs is sensitive to many modelling factors such as the
number of input nodes and the training sample size (Zhang and Hu
\protect\hyperlink{ref-zhang1998}{1998}).

\subsection{Fuzzy Logistic Systems}\label{fuzzy-logistic-systems}

Fuzzy Logistic Systems were initially developed to solve problems
involving linguistic terms, and have been used successful in financial
forecasting (Khashei, Bijari, and Ardali
\protect\hyperlink{ref-khashei2009}{2009}). Unlike computers, the human
decision-making process includes a range of possibilities spanning `yes'
and `no'. Fuzzy logic tries to imitate this reasoning and
decision-making process by taking a discrete decision space that
consists of a large number of states and turning it into a continuous
space. This allows for more finer decisions to be provided as opposed to
discrete decisions.

Santos, Costa, and Santos Coelho
(\protect\hyperlink{ref-santos2007}{2007}) investigated how well fuzzy
systems and neural networks perform compared to the traditional ARMA and
GARCH model. He examined the forecasts of Brazilian exchange rate
returns looking at different frequencies of the series and compared
their one-step-ahead forecasts. By analysing the RMSE, U-Theil
inequality index, percentage of corrected predicted signals (CPS), and
the Pesaran-Timmermann (PT) predictive failure statistic, he found that
the Fuzzy Systems and Neural Networks achieved higher returns based on
the forecasts they produced. Similar results were found by Khashei,
Bijari, and Ardali (\protect\hyperlink{ref-khashei2009}{2009}) when he
analysed the predictive capabilities of Fuzzy Systems, Neural Networks,
the traditional ARIMA model, and a Fuzzy ARIMA model. The MAE and MSE of
the ARIMA model were higher compared to the other models.

Although Fuzzy Logistic Systems has the advantage over ARIMA models that
they can be applied to data with few observations available, it gives
acceptable rather than accurate reasoning and are more suitable for
problems which do not require high accuracy. Like ANN's there is no
systematic approach to designing Fuzzy Logistic Systems and they are
only understandable when simple.

\section{Hybrid ARIMA Models}\label{hybrid-arima-models}

Over time, many researchers began to think of ways in which to harness
the advantages of tractable linear models such as the ARIMA, and of more
complex non-linear models. By combining the ARIMA model with other
forecasting methods, the advantages of both forecasting methods are
leveraged while simultaneously improving their limitations. Some of the
hybrid models which have been commonly used are the Fuzzy ARIMA and
ANN-ARIMA model.

\subsection{Fuzzy ARIMA}\label{fuzzy-arima}

The ARIMA model produces very accurate forecasts over short time
horizons however it has the limitation of requiring more than 100
observations of historical data (Tseng et al.
\protect\hyperlink{ref-tseng2001}{2001}). In a world that is constantly
changing and with the rapid advancement of technology, access to large
amounts of historical data is difficult to obtain. On the contrary, a
fuzzy regression model requires little historical data however produces
wide prediction intervals if extreme values are present in the data. The
Fuzzy ARIMA model combines the ARIMA and fuzzy regression model to
exploit the advantages of both models while simultaneously overcoming
their limitations. Tseng et al.
(\protect\hyperlink{ref-tseng2001}{2001}) proposed applying a Fuzzy
ARIMA model to forecast the exchange rate of Taiwan dollars to US
dollars to demonstrate the model's appropriateness and power. The Fuzzy
ARIMA not only produced forecasts that were superior to the ARIMA and
fuzzy time series models but also provided an upper and lower bound
which can be used by decision makers to determine the best and worst
possible situations.

\subsection{ANN-ARIMA}\label{ann-arima}

Although ARIMA models are powerful, they require non-stationary data to
be differenced and impose prior assumptions onto the distribution of the
data (Ince and Trafalis \protect\hyperlink{ref-ince2006}{2006}). In
contrast, machine learning techniques such as ANNs do not impose any
assumptions onto the data generating process however, being data-driven,
are sensitive to the number of input nodes used. The ANN-ARIMA model
draws on the strengths of both models and overcomes these individual
difficulties. Ince and Trafalis (\protect\hyperlink{ref-ince2006}{2006})
created an ANN-ARIMA model by using the ARIMA to determine the number
input nodes required by an ANN for three exchange rates. When the
ANN-ARIMA model was compared to the pure ARIMA, the hybrid model
outperformed the ARIMA based on the MSE. The findings of Khashei and
Bijari (\protect\hyperlink{ref-khashei2011}{2011}) study agree with
these results. They implemented an ARIMA model and used its residuals
together with past observations of the data as inputs for the ANN. Their
hybrid model had superior in-sample and out-sample forecasts compared to
the random walk, linear AR and ANN model.

Although the traditional ARIMA models produce less superior forecasts
than its hybrid forms and other complex non-linear techniques, its
forecasts are still satisfactory. They are simple models that are easy
to implement and have a consistent method of model design and selection.
ARIMA models are also more robust and efficient than complex structural
models in relation to short-run forecasting. The fact that they have
been used as the foundation for more advanced models and have commonly
been used as a benchmark for comparison justifies it as a good starting
point to compare it to Facebook's forecasting method, Prophet, that was
recently released.

\section{Forecasting with Prophet}\label{forecasting-with-prophet}

The techniques that have been considered for exchange rate forecasting
thus far, require the analyst to have vocational knowledge about time
series. Prophet differs to traditional time series models in that it is
flexible and can be customised by a large number of non-experts who have
little knowledge about time series, however have domain knowledge about
the data generating process. Prophet allows for a large number of
forecasts to be produced across a variety of problems and consists of a
robust evaluation system that allows for a large number of forecasts be
evaluated and compared. This is Facebook's definition of forecasting at
scale.

\subsection{Construction of Prophet}\label{construction-of-prophet}

Prophet consists of a decomposable model of the form:

\[ 
    y(t) = g(t) + s(t) + h(t) + e_t
\] where the components of the model represent the growth, seasonality
and holiday respectively, and \(e_t\) is white noise.

These components consist of linear and non-linear functions of time.
This differs to ARIMA models in which future values of the process are
linear functions of previous observations and lagged residuals. Prophet
is more like a Generalized Additive Model (GAM) which is a regression
model that consists of non-linear and linear regression functions
applied to predictor variables (Taylor and Letham
\protect\hyperlink{ref-taylor2017}{2017}). Prophet, like the GAM, frames
the forecasting problem as a curve fitting exercise, using backfitting
to find the regression functions. The GAM is fitted quickly, allowing
the analyst to interactively change the model parameters (Taylor and
Letham \protect\hyperlink{ref-taylor2017}{2017}).

The growth component is modelled in a similar way to population growths
which use a logistic growth model (Taylor and Letham
\protect\hyperlink{ref-taylor2017}{2017}). Populations typically grow
non-linearly (although the growth component could also be linear) up to
an upper bound known as the carrying capacity, and remains constant
thereafter. The rate at which the population grows changes over time,
and this is accounted for by including changepoints in the model where
the growth rate can be automatically selected and may be adjusted
(Taylor and Letham \protect\hyperlink{ref-taylor2017}{2017}). This
allows non-experts with knowledge about events that may affect growth to
use the parameter as a knob and adjust it to increase or decrease the
number of changepoints (Taylor and Letham
\protect\hyperlink{ref-taylor2017}{2017}). It also allows for the
analyst to add changepoints which the automatic selection procedure may
have missed (Taylor and Letham
\protect\hyperlink{ref-taylor2017}{2017}). Furthermore, analysts may
also specify the carrying capacity and adjust it based on their
knowledge of the total market size (Taylor and Letham
\protect\hyperlink{ref-taylor2017}{2017}).

The decomposable form of the model allows for components to be easily
added to it. This allows for multiple seasonality components with
different periods to be added to the model. The variance of the
parameters of the seasonality component's distribution can be adjusted
by analysts to smooth the model and change how much of historical
seasonality is projected to the future (Taylor and Letham
\protect\hyperlink{ref-taylor2017}{2017}).

The name, date, and country of past and future holidays and events may
also be inputted by the analyst into a list (Taylor and Letham
\protect\hyperlink{ref-taylor2017}{2017}). By specifying the country in
which the events take place or the holidays occur, separate lists can be
populated for global events/holidays and country-specific
events/holidays. The union of the two lists are then used for
forecasting. Like seasonality, the variance of the parameters of the
holiday component's distribution, can be adjusted by analysts to smooth
the model (Taylor and Letham \protect\hyperlink{ref-taylor2017}{2017}).

\subsection{Empirical Applications}\label{empirical-applications-1}

Prophet's Bayesian approach to forecasting allows the analyst to
incorporate their expert knowledge in the model building process and has
produced significantly improved forecasts compared to the ARIMA model.
Taylor and Letham (\protect\hyperlink{ref-taylor2017}{2017}) forecasted
the number of events on Facebook using Prophet. The time series was
impacted by holidays, had strong multi-period seasonality, and a
piecewise trend. The forecasts produced by Prophet were compared to
common automatic forecasting techniques such as exponential smoothing,
ARIMA, and the seasonal naïve model, as well as to simple models such as
the naïve model. While exponential smoothing and the seasonal naïve
model were quite robust, the ARIMA forecasts were fragile. No model
besides Prophet accounted for the dips around holidays and the upward
trend of the time series towards later observations.

\subsection{Semi-Automatic Selection}\label{semi-automatic-selection}

When a large number of forecasts are produced, manually identifying
problematic forecasts becomes a time consuming and difficult task.
Prophet provides a semi-automated forecast evaluation system that
selects the best model that fits the data. When there are large forecast
errors, the forecasts are flagged so that the analyst can explore the
cause of the errors, identify and remove potential outliers and either
adjust the model or choose a more appropriate model (Taylor and Letham
\protect\hyperlink{ref-taylor2017}{2017}). Unlike ARIMA's
fully-automated evaluation system, Prophet provides interactive feedback
and keeps the analyst in the loop.

Prophet's ability to forecast at scale enables it to model a wide
variety of data and may be able to capture the non-linearity present in
exchange rates. In contrast, the ARIMA model fails to capture the
non-linearity inherent in exchange rate data and only produces good
forecasts over short time horizons. Prophet is also a flexible model
that has intuitive parameters which can be easily interpreted and
modified by human beings (Taylor and Letham
\protect\hyperlink{ref-taylor2017}{2017}). This may be useful to
non-experts who have domain knowledge about factors that may affect the
movement of exchange rate prices. Furthermore, unlike the ARIMA model,
Prophet can produce forecasts over irregular time intervals and allows
for missing values in the time series without the need for interpolation
(Taylor and Letham \protect\hyperlink{ref-taylor2017}{2017}). If Prophet
produces forecasts that are as good as the ARIMA model when forecasting
Bitcoin/ZAR, it can be compared to more complex forecasting methods that
are less tractable such as the hybrid models and machine learning
techniques seen earlier.

\section{Mincer-Zarnowitz Approach to Forecast
Evaluation}\label{mincer-zarnowitz-approach-to-forecast-evaluation}

The Mincer-Zarnowitz approach to evaluating forecast accuracy is
commonly used and can be useful when comparing the forecasts produced by
Prophet and the ARIMA model. Mincer and Zarnowitz
(\protect\hyperlink{ref-mincer1969}{1969}) proposed an absolute and
relative measure to evaluate forecast accuracy. Absolute measures
consider the distance between actual and predicted values. To analyse
the absolute errors produced by the forecasts, the observed values are
regressed against the predicted values. The intercept of the regression
equation represents the mean distance between the observed and predicted
values while the slope represents the correlation between the residual
errors and the predicted values. A zero intercept implies that the
forecasts are unbiased and do not overestimate or underestimate the
data, while a unity slope implies that the forecasts are efficient and
uncorrelated with the residual errors. A joint hypothesis test is
performed to check this efficiency and bias of the forecasts, and models
which produce the best results are selected.

Absolute forecast measures cannot be used to make comparisons between
forecasts with different scales or economic variables. Furthermore, the
size of forecasting errors is not as significant as the consequences of
forecasting errors and how they impact the decision making process.
Hence, rather than using absolute measures of forecast accuracy, Mincer
and Zarnowitz (\protect\hyperlink{ref-mincer1969}{1969}) suggested the
use of relative measures. Relative accuracy analysis allows for
meaningful comparisons of different forecasting methods to be made. It
uses an index which considers the ratio of the MSE of the forecast to
the MSE of a benchmark forecasting method. A useful benchmark that may
be used is an extrapolation of the data history, as it is a cost
effective and accessible method, however any method which is relevant
for comparison may be used. This ratio is known as the Relative Mean
Square Error (RM) for forecast evaluation. The numerator can be viewed
as a return which is inversely proportional to the MSE error of the
forecasts, while the denominator can be viewed as the cost of
forecasting which is inversely proportional to the MSE of the benchmark
(Mincer and Zarnowitz \protect\hyperlink{ref-mincer1969}{1969}). Hence
the ratio is representative of a rate of return index and ranks the
performance of forecasts as such. Models that result in forecasts with a
RM that is less than one are said to produce superior forecasts to the
benchmark model being considered.

\section{Conclusion}\label{conclusion}

ARIMA models have been highly popular since its introduction by Box and
Jenkins and are widely used for financial forecasting. They are easy to
implement, have a systematic procedure for model design, and produce
good forecasts over short time horizons. Although robust, they have the
limitation of requiring large amounts of historical observations and
they fail to capture the non-linearity inherent in exchange rate data.
With the evolution of computational power and the advancement of
statistics, more complex techniques such as ANN's, Fuzzy Logistic
Systems, and Hybrid ARIMA models have been proposed and compared to the
traditional ARIMA model. Although these models produce superior results
to the ARIMA model, they are less tractable and require the analyst to
understand the complex statistical intricacies involved. Prophet's
Bayesian approach to forecasting is flexible and allows for non-experts
to incorporate their domain knowledge in the model building process. If
Prophet produces forecasts that are as good as the ARIMA model when
forecasting Bitcoin/ZAR, it can be compared to these more complex
forecasting methods that are less tractable and scalable.

\newpage

\section*{References}\label{references}
\addcontentsline{toc}{section}{References}

\hypertarget{refs}{}
\hypertarget{ref-abu1996}{}
Abu-Mostafa, Yaser S, and Amir F Atiya. 1996. ``Introduction to
Financial Forecasting.'' \emph{Applied Intelligence} 6 (3). Springer:
205--13.

\hypertarget{ref-anastasakis2009}{}
Anastasakis, Leonidas, and Neil Mort. 2009. ``Exchange Rate Forecasting
Using a Combined Parametric and Nonparametric Self-Organising Modelling
Approach.'' \emph{Expert Systems with Applications} 36 (10). Elsevier:
12001--11.

\hypertarget{ref-castillo2002}{}
Castillo, Oscar, and Patricia Melin. 2002. ``Hybrid Intelligent Systems
for Time Series Prediction Using Neural Networks, Fuzzy Logic, and
Fractal Theory.'' \emph{IEEE Transactions on Neural Networks} 13 (6).
IEEE: 1395--1408.

\hypertarget{ref-cermak2017}{}
Cermak, Vavrinec. 2017. ``Can Bitcoin Become a Viable Alternative to
Fiat Currencies? An Empirical Analysis of Bitcoin's Volatility Based on
a Garch Model.''

\hypertarget{ref-cheung1993}{}
Cheung, Yin-Wong. 1993. ``Long Memory in Foreign-Exchange Rates.''
\emph{Journal of Business \& Economic Statistics} 11 (1). Taylor \&
Francis: 93--101.

\hypertarget{ref-etuk2012}{}
Etuk, Ette Harrison, and others. 2012. ``A Seasonal Arima Model for
Daily Nigerian Naira-Us Dollar Exchange Rates.'' \emph{Asian Journal of
Empirical Research} 2 (6). Asian Economic; Social Society: 219--27.

\hypertarget{ref-etuk2013}{}
Etuk, Ette Harrison, Dagogo SA Wokoma, and Imoh Udo Moffat. 2013.
``Additive Sarima Modelling of Monthly Nigerian Naira-Cfa Franc Exchange
Rates.'' \emph{European Journal of Statistics and Probability} 1 (1):
1--12.

\hypertarget{ref-fahimifard2009}{}
Fahimifard, SM, Masuod Homayounifar, M Sabouhi, and AR Moghaddamnia.
2009. ``Comparison of Anfis, Ann, Garch and Arima Techniques to Exchange
Rate Forecasting.'' \emph{Journal of Applied Sciences} 9 (20): 3641--51.

\hypertarget{ref-franses1998}{}
Franses, Philip Hans, and Paul Van Homelen. 1998. ``On Forecasting
Exchange Rates Using Neural Networks.'' \emph{Applied Financial
Economics} 8 (6). Taylor \& Francis: 589--96.

\hypertarget{ref-spyros1997}{}
Hibon, Michael, and Spyros Makridakis. 1997. ``ARMA Models and the
Box--Jenkins Methodology.'' John Wiley \& Sons, Ltd.

\hypertarget{ref-ince2006}{}
Ince, Huseyin, and Theodore B Trafalis. 2006. ``A Hybrid Model for
Exchange Rate Prediction.'' \emph{Decision Support Systems} 42 (2).
Elsevier: 1054--62.

\hypertarget{ref-kamruzzaman2003}{}
Kamruzzaman, Joarder, and Ruhul A Sarker. 2003. ``Forecasting of
Currency Exchange Rates Using Ann: A Case Study.'' In \emph{Neural
Networks and Signal Processing, 2003. Proceedings of the 2003
International Conference on}, 1:793--97. IEEE.

\hypertarget{ref-khashei2011}{}
Khashei, Mehdi, and Hehdi Bijari. 2011. ``Exchange Rate Forecasting
Better with Hybrid Artificial Neural Networks Models.'' \emph{Journal of
Mathematical and Computational Science} 1 (1). Science \& Knowledge
Publishing Corporation Limited (SCIK): 103.

\hypertarget{ref-khashei2009}{}
Khashei, Mehdi, Mehdi Bijari, and Gholam Ali Raissi Ardali. 2009.
``Improvement of Auto-Regressive Integrated Moving Average Models Using
Fuzzy Logic and Artificial Neural Networks (Anns).''
\emph{Neurocomputing} 72 (4). Elsevier: 956--67.

\hypertarget{ref-khashei2012}{}
---------. 2012. ``Hybridization of Autoregressive Integrated Moving
Average (Arima) with Probabilistic Neural Networks (Pnns).''
\emph{Computers \& Industrial Engineering} 63 (1). Elsevier: 37--45.

\hypertarget{ref-kuan1995}{}
Kuan, Chung-Ming, and Tung Liu. 1995. ``Forecasting Exchange Rates Using
Feedforward and Recurrent Neural Networks.'' \emph{Journal of Applied
Econometrics} 10 (4). Wiley Online Library: 347--64.

\hypertarget{ref-lin2012}{}
Lin, Chiun-Sin, Sheng-Hsiung Chiu, and Tzu-Yu Lin. 2012. ``Empirical
Mode Decomposition--based Least Squares Support Vector Regression for
Foreign Exchange Rate Forecasting.'' \emph{Economic Modelling} 29 (6).
Elsevier: 2583--90.

\hypertarget{ref-mincer1969}{}
Mincer, Jacob A, and Victor Zarnowitz. 1969. ``The Evaluation of
Economic Forecasts.'' In \emph{Economic Forecasts and Expectations:
Analysis of Forecasting Behavior and Performance}, 3--46. NBER.

\hypertarget{ref-nwankwo2014}{}
Nwankwo, Steve C. 2014. ``Autoregressive Integrated Moving Average
(Arima) Model for Exchange Rate (Naira to Dollar).'' \emph{Academic
Journal of Interdisciplinary Studies} 3 (4): 429.

\hypertarget{ref-ruppert2015}{}
Ruppert, David, and David S Matteson. 2015. \emph{Statistics and Data
Analysis for Financial Engineering: With R Examples}. Springer.

\hypertarget{ref-santos2007}{}
Santos, André Alves Portela, Newton Carneiro Affonso da Costa, and
Leandro dos Santos Coelho. 2007. ``Computational Intelligence Approaches
and Linear Models in Case Studies of Forecasting Exchange Rates.''
\emph{Expert Systems with Applications} 33 (4). Elsevier: 816--23.

\hypertarget{ref-taylor2017}{}
Taylor, Sean J, and Benjamin Letham. 2017. ``Forecasting at Scale.''

\hypertarget{ref-tseng2001}{}
Tseng, Fang-Mei, Gwo-Hshiung Tzeng, Hsiao-Cheng Yu, and Benjamin JC
Yuan. 2001. ``Fuzzy Arima Model for Forecasting the Foreign Exchange
Market.'' \emph{Fuzzy Sets and Systems} 118 (1). Elsevier: 9--19.

\hypertarget{ref-zhang1998}{}
Zhang, Gioqinang, and Michael Y Hu. 1998. ``Neural Network Forecasting
of the British Pound/Us Dollar Exchange Rate.'' \emph{Omega} 26 (4).
Elsevier: 495--506.

% Force include bibliography in my chosen format:
\newpage
\nocite{*}
\bibliography{}





\end{document}
